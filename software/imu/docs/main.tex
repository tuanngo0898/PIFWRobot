\documentclass[oneside, 12pt, a4paper]{book}

\usepackage[utf8]{inputenc}
\usepackage[english]{babel}

\usepackage{amsmath}
\usepackage{amsfonts}
\usepackage{amssymb}

\providecommand{\main}{.}

\begin{document}
\chapter{EKF}

\section{Kalman filter}
The simplest of state space models are linear models, which can be expressed with equations of the following form.

\begin{equation}
    \begin{split}
        \mathbf{x}_t &= \mathbf{F}\mathbf{x}_{t-1}+\mathbf{w}_x\\
        \mathbf{z}_t &= \mathbf{H}\mathbf{x}_t+\mathbf{w}_z
     \end{split}
\end{equation}
where
\begin{itemize}
    \item $\mathbf{x}_t \in \mathbb{R}^n$ is the state of the system describing the condition of n elements at time $t$. 
    \item $\mathbf{z}_t \in \mathbb{R}^m$ are the measurements at time $t$.
    \item $\mathbf{w}_x \thicksim \mathcal{N}(0,\mathbf{Q}_t)$ is the process noise at time $t$.
    \item $\mathbf{w}_z \thicksim \mathcal{N}(0,\mathbf{R}_t)$ is the measurement noise at time $t$.
    \item $\mathbf{F}$ is called either \textbf{State Transition Matrix} or \textbf{Fundamental Matrix}.
    \item $\mathbf{H}$ is the measurement model matrix.
\end{itemize}

\begin{equation}
    x = x_0 + \dot{x}*t
\end{equation}
\begin{equation}
    \dot{x} = (-1/t)x_0 + (1/t)x
\end{equation}

\begin{equation}
    \begin{split}
        x_1 &= x_0 + v_x t \\
        x_2 &= x_0 + v_x t
    \end{split}
\end{equation}

\begin{equation}
    x_1 + x_2 = 2*x_0 + 2v_x t 
\end{equation}

\begin{equation}
    x_3 = x_0 + 2v_x t
\end{equation}

\end{document}